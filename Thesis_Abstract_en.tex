\thispagestyle{empty} 
%\geometry{top=3cm,right=2.5cm,bottom=2.5cm,left=3.5cm} 

\begin{latin}
\centerline{\textbf{\large{Abstract}}}
\begin{quote}
	
Lots of essential structures can be modeled as sequences and sequences can be utilized to model the structures like molecules, graphs and music notes. On the other hand, generating meaningful and new sequences is an important and practical problem in different applications. Natural language translation and drug discovery are examples of sequence generation problem. However, there are substantial challenges in sequence generation problem. Discrete spaces of the sequence and challenge of the proper objective function can be pointed out. 

On the other, the baseline methods suffer from issues like exposure bias between training and test time, and the ill-defined objective function. So, the necessity of new methods is available.

In recent years, there has been development in image generation by usage of generative adversarial networks (GANs). These networks have attracted attention for sequence generation lately, but since sequences are discrete, this cannot be done easily, and new approaches like reinforcement learning and approximation should be utilized. Furthermore, the problem of instability in generative adversarial networks causes new challenges.

In this research, based on the idea of generative adversarial models, a new iterative method is proposed for sequence generation problem, such that in every step of the algorithm, the model is trained against itself by using the generated samples. 
The idea of the proposed method is based on the ratio estimation which enables the model to overcome the problem of discreteness in data. Also, the proposed method is more stable than the other GAN-based methods. It also should be noted that the exposure bias problem does not exist in the proposed method.

Since the evaluation of the generated sequences is also an essential challenge in the field of sequence generation, we reviewed the evaluation criterion for sequence generation and also proposed three new methods for evaluating the new sequences which in contrast to previous criterions, examines both the quality and diversity of the new samples.

Experiments show the superiority of the proposed method to previous methods.


\vskip 1cm
\textbf{Keywords:} \textit{Sequence Generation, Adversarial Networks, Neural Network, Deep Learning}
\end{quote}
\end{latin}

